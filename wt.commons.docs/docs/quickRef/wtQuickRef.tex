\documentclass[10pt, twocolumn]{article}

\usepackage{listings}
\usepackage{multicol}
\usepackage{fullpage}
\usepackage[hmargin=1cm,vmargin=2cm]{geometry}

\newcommand{\topic}[1]{
\hRule

{\sf{\textbf{#1}}}
\vspace{0.1in}

%\hRule
}
\newcommand{\code}[1]{\lstinline[basicstyle=\small\ttfamily]{#1}}
\newcommand{\hRule}{\rule{\linewidth}{0.2mm}}

\newcommand{\Indent}{\hspace{0.2in}}


\newcommand{\desc}[1]{\Indent\parbox[b]{3.5in}{#1}}


%------------------------------------------------------------------------------
\begin{document}
%------------------------------------------------------------------------------
%\begin{multicols}{2}

\thispagestyle{empty} % suppress page #s


%------------------------------------------------------------------------------
{
\Large{\sf\textit{WindowTester API Quick Reference}}
}

%------------------------------------------------------------------------------


%------------------------------------------------------------------------------
%------------------------------------------------------------------------------
\topic{IUIContext methods}
%------------------------------------------------------------------------------

\code{ILocator click(ILocator locator)}

\desc{click the widget identified by this locator}

\code{IWidgetLocator find(ILocator locator)}

\desc{find the widget identified by this locator}

\code{IWidgetLocator[] findAll(ILocator locator)}

\desc{find all the widgets identified by this locator}

\code{void wait(ICondition condition)}

\desc{wait for this condition to become true}

\code{void assertThat(ICondition condition)}

\desc{assert that this condition is true}

%\code{IConditionMonitor getConditionMonitor()}
%
%\desc{get the current condition monitor}


%------------------------------------------------------------------------------
%------------------------------------------------------------------------------
\topic{Locators}
%------------------------------------------------------------------------------

\code{new TreeItemLocator(``General/Project'')}

\desc{the tree item ``Project'' with parent ``General''}

\code{new TreeItemLocator(``(General|Simple)/Project'')}

\desc{the tree item ``Project'' with parent ``General'' or ``Simple''}

\code{new TreeItemLocator(``Project'',}
\indent\indent\indent\code{new ViewLocator(``org.acme.Explorer''))}

\desc{the tree item ``Project'' in the ``org.acme.Explorer'' view}

\code{new MenuItemLocator(``Window/&Preferences(...)?'')}

\desc{the preferences menu (with or without trailing dots)}



%------------------------------------------------------------------------------
%------------------------------------------------------------------------------
\topic{Conditions}
%------------------------------------------------------------------------------

\code{ui.wait(new ShellShowingCondition(``New Project''))}

\desc{wait for the ``New Project'' shell to be showing}

\code{ui.wait(new ViewLocator(``JUnit'').isVisible()))}

\desc{wait for the JUnit view to be visible}

\code{ui.wait(new ProjectExistsCondition(``org.acme.MyProject''))}

\desc{wait for the ``org.acme.MyProject'' to exist}

\code{ui.wait(new JobsCompleteCondition())}

\desc{wait for the all the Eclipse Jobs to be complete}

%\code{ui.wait(new ICondition() \{} \\
%\indent\indent\code{public boolean test() \{} \\
%\indent\indent\indent\code{return something();} \\
%\indent\indent\code{\}}\\
%\indent\code{\})}
%
%\desc{wait for \textit{something} to be true}


%------------------------------------------------------------------------------
%------------------------------------------------------------------------------
\topic{Assertions}
%------------------------------------------------------------------------------


\code{ui.assertThat(new ButtonLocator(``OK'').isEnabled())}

\desc{assert that the ``OK'' button is enabled}

\code{ui.assertThat(new EditorLocator(``Foo.java'').isActive())}

\desc{assert that the editor open on ``Foo.java'' is active}


%------------------------------------------------------------------------------
%------------------------------------------------------------------------------
\topic{Condition Monitors}
%------------------------------------------------------------------------------

\code{IShellMonitor sm =} \\
\indent\indent\indent\code{(IShellMonitor)ui.getAdapter(IShellMonitor.class)}
\indent\code{sm.add(new PerspectiveSwitchedShellHandler())}

\desc{register a handler that clicks the ``No'' button on the
``Confirm Perspective Switch'' Shell.}



%------------------------------------------------------------------------------
%------------------------------------------------------------------------------
\topic{Naming}
%------------------------------------------------------------------------------

\code{button.setData(``name'', ``named.button'')}

\desc{give the button the name ``named.button''}

\code{ui.find(new NamedWidgetLocator(``named.widget''))}

\desc{find the ``named.button'' button at runtime}


%------------------------------------------------------------------------------
%------------------------------------------------------------------------------
\topic{Widget Access}
%------------------------------------------------------------------------------

\code{IWidgetLocator wl = ui.find(new ButtonLocator(``OK''))}

\desc{find the ``OK'' button}

\code{final Button b = (Button)((IWidgetReference)wl).getWidget()}

\desc{get the underlying widget}

\code{final boolean[] result = new boolean[1];}
\indent\code{Display.getDefault().syncExec(new Runnable() \{} \\
\indent\indent\code{public void run() \{} \\
\indent\indent\indent\code{result[0] = button.getAlignment();} \\
\indent\indent\code{\}}\\
\indent\code{\});} \\
\indent\code{assertEquals(SWT.UP, result[0]);}

\desc{access the alignment property safely on the UI thread}



%------------------------------------------------------------------------------
%------------------------------------------------------------------------------
\topic{Constants}
%------------------------------------------------------------------------------



%------------------------------------------------------------------------------
%------------------------------------------------------------------------------

%\end{multicols}
%------------------------------------------------------------------------------



\end{document} 